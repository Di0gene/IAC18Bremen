\section{Conclusions}
\label{sec:conclusions}
The mission and system design is not expressly innovative and unique, but it has been performed by a clever usage of both the present (EP) and the classical (CP) paths. 
The following highlights the critical aspects of the hybrid strategy in the context of the findings of this study.
\begin{enumerate}[label=(\alph*)]
\item The growth in signal throughput of the Communication satellites in GEO is fundamental in developing a valuable strategy for Hybrid missions.
\item Dual Stage Spacecraft for Hybrid Transfer (Figure \ref{fig:dualstageconfiguration}) is mass-efficient and time-of-flight-efficient, but the overall design is complex because of the jettisoning phase of the CPM that requires a deep analysis.
\item The mass of the CPM jettisoned from the spacecraft is comparable to the one of a small-size satellite, so reducing the complexity for the controlled/uncontrolled re-entry of the CPM or its removal towards the Graveyard Orbit.
\item Hybrid Transfer is the family of solutions, within the launch-mass vs transfer time framework, where the FCT and the FET are only the boundaries of the "best transfers" domain, as shown in \figurename\ref{fig:paretomasstime}.
\item Hybrid propulsion strategy solves the problem of multi-spiral, low-thrust transfer of SEP only missions, allowing the reduction of the time spent in the Van Allen Belt.
\item When compared with the FCT, the Hybrid Transfer is considerably convenient because the cost of mission in \figurename\ref{fig:paretocosttime} is greatly decreased and attractive transfer times are provided ($\sim 20-60$ days).
%\item Performance (mass) of the launcher plays a fundamental role for mission costs, as outlined in Figure (\ref{fig:costmissionSO}). 
Furthermore, Pareto-front for mission costs brings to light that the FET does not belong to the set of the best solutions, indeed, there are transfer paths which deliver the payload in GEO in a bit more of half FET time-of-flights for a bit lower mission cost.
This observation is useful dealing with the development of future satellites because \textbf{maybe}, (and the author emphasizes \emph{maybe}), an all-electric platform to reach GEO is not the best solution to achieve the mass-saving and economic revenue purposes simultaneously.
\item Throttling performance for SEP thrusters is fundamental requirement to design a flexible and efficient GEO platform.% Moreover, if properly designed, the spacecraft may be neither be oversized nor undersized at a system level (see Section \ref{subsec:design}).
\item The Hybrid Transfer solution does not demand radical development of new components of propulsive systems, and the study has shown that typical propulsion architectures may be implemented into the hybrid spacecraft.
%\item Changes of the propulsion systems technologies selected (\textsc{cp} and \textsc{ep} systems components), as well as the starting orbit or mission requirements, can be easily implemented within the algorithm. Thus, suitable versatility capabilities are provided.
\item For what concerns future works, dynamic model may be yield more realistic through the introduction of the Earth-Shadow eclipse period during the EP transfer, the cost for the SEP components shall be updated with off-the-shelves items, and the issues arisen from the CPM jettisoning may be dealt with in details.
\end{enumerate}