%!TEX root = ../main.tex
%!TEX spellcheck = en-en_US

\normalsize
\iabstract{
The Earth--Moon system is constantly bombarded by meteoroids of different sizes, and their numbers are significant. The Lunar Meteoroid Impacts Observer (LUMIO) is a CubeSat mission that shall observe, quantify, and characterize the meteoroid impacts by detecting their flashes on the lunar farside. This complements the knowledge gathered by Earth-based observations of the lunar nearside, thus synthesizing a global information on the lunar meteoroid environment. A spaced-based asset improves the quality and quantity of lunar meteoroid impact flashes detection and helps initiate a Lunar Situational Awareness program. LUMIO is winner of ESA's LUCE (Lunar CubeSat for Exploration) SysNova competition, and as such it is being considered by the Agency for implementation in the near future. In this paper, an orthodox trade-off analysis is carried out for LUMIO operative orbit selection, ranging from selenocentric Keplerian orbits to periodic orbits of the Earth--Moon restricted three-body problem, and motion in a quasi-real solar system model. LUMIO is planned to be released on an elliptical quasi-polar lunar orbit. We propose the implementation of a sophisticated orbit design, concept of operations, and station-keeping strategy: LUMIO is placed on a quasi-halo orbit about Earth--Moon $L_2$ where permanent full-disk observation of the lunar farside is made. This prevents having background noise due to Earthshine, and thus permits obtaining high-quality scientific products. Repetitive operations are also foreseen, the orbit being in near $2$:$1$ resonance with the Moon orbit. A large set of quasi-halo orbit is computed in the high-fidelity Roto-Pulsating Restricted $n$-Body Problem. LUMIO operative orbit is then selected upon minimization of station-keeping cost and transfer cost. In this work, we show a comprehensive orbit design for LUMIO and discuss possible improvements in view of the mission implementation.
}
