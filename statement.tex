\section{Statement of the Problem}
\label{statement}

\subsection{Dynamic Model}

The classical Restricted Two-Body Problem is used in the remainder to model the motion of the spacecraft around the primary, properly modified to take into account the effect of the control effort. The system of differential equations reads:

\begin{equation}
\mathbf{\dot{x}} = \mathbf{f}(\mathbf{x}, \boldsymbol{\alpha}, u, t)
\quad
\Rightarrow
\quad
\begin{bmatrix}
\mathbf{\dot{r}} \\ \mathbf{\dot{v}} \\ \dot{m}
\end{bmatrix}
=
\begin{bmatrix}
\mathbf{v} \\
-\frac{\mu}{\parallel \mathbf{r}^3 \parallel} \mathbf{r} + \frac{T}{m} \boldsymbol{\alpha} u \\
- \frac{T}{I_{sp} g_0} u
\end{bmatrix}
\end{equation}

Where $\mathbf{r} = [ x~y~z]^T$ is the position vector, $\mathbf{v} = [v_x~v_y~v_z]^T$ the velocity vector, $m$ the spacecraft mass, $\mu$ the gravitational constant, $T_{max}$ is the engine thrust, $\boldsymbol{\alpha}$ the thrust direction, $I_{sp}$ the specific impulse, $g_0$ the gravity acceleration and $u$ the throttle factor. This latter has a twofold nature, and discerns whether the engine is on ($f=1$) or off ($f=0$).

According to Optimal Control Theory \cite{bryson}, optimization problem consists in finding the control law which minimizes a certain Cost Function, which in Fuel-Optimal problems reads: 

\begin{equation}
J = \int_{t_0}^{t_f} \mathcal{L}~dt = \int_{t_0}^{t_f} \frac{T}{I_{sp} g_0} u~dt
\end{equation}

Where $\mathcal{L}$ is called performance index, and $t_0$ and $t_f$ are the initial and final time, respectively. The aforementioned duplex nature of the throttle factor $u$ introduces discontinuities within the equations of motion which make harsh their integration;. this fact has been taken into account and faced through a common approach that relies on the so-called smoothing, which in turns modifies the cost function and allows shifting from a discontinuous problem to a smoother one:

\begin{equation}
J = \frac{T}{I_{sp} g_0}\int_{t_0}^{t_f}  u \left[ 1 - \varepsilon(1-u)\right]~dt
\end{equation}

Where $\varepsilon$ will henceforth be called homotopy factor. Thus, when it equates the unity $(\varepsilon=1$), a quadratic performance index is obtained. The fulfillment of dynamic model is mandatory along the whole trajectory and, to this aim, the system of differential equations is adjoined to the cost function:

\begin{equation}
\tilde{J} = \int_{t_0}^{t_f}\left\lbrace \frac{T_{max}}{I_{sp} g_0} u + \boldsymbol{\lambda}(t) \cdot \left[ \mathbf{f}(\mathbf{x}, \boldsymbol{\alpha},u ) - \mathbf{\dot{x}} \right] \right\rbrace dt
\end{equation}

Where the vector $\boldsymbol{\lambda}$ is the Lagrange multipliers vector. 

\begin{equation}
\quad
\Rightarrow
\quad
\mathcal{H} = \frac{T_{max}}{I_{sp} g_0} f+ \boldsymbol{\lambda} \cdot \mathbf{f}
\end{equation}



\begin{equation}
f
=
\begin{cases}
1 & \mathcal{S} < 0 \\
0 & \mathcal{S} > 0 \\
\end{cases}
\end{equation}

\begin{equation}
f
=
\begin{cases}
1 & \mathcal{S} < 0 \\
0 & \mathcal{S} > 0 \\
\frac{\varepsilon - \mathcal{S}}{2 \varepsilon} & |\mathcal{S}| < \varepsilon \\
\end{cases}
\end{equation}

Pontryagin's Maximum Principle is applied for 

\begin{equation}
\boldsymbol{\alpha} = \mathbf{p} = -\frac{\boldsymbol{\lambda_v}}{\parallel \boldsymbol{\lambda_v} \parallel}
\end{equation}

\begin{equation}
\begin{bmatrix}
\mathbf{\dot{x}} \\ \boldsymbol{\dot{\lambda}} 
\end{bmatrix}
=
\mathbf{F}(\mathbf{x}, \boldsymbol{\lambda}, \mathbf{u})
=
\begin{bmatrix}
\frac{\partial \mathcal{H}}{\partial \boldsymbol{\lambda}} \\
-\frac{\partial \mathcal{H}}{\partial \mathbf{x}}
\end{bmatrix}
\end{equation}

Integration scheme follows the flowchart of \cite[Figure 1]{paper_topputo}

\subsection{Time-Optimal Problem}

