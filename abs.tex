%!TEX root = main.tex

%% here You can define Your macros
\normalsize
%% your abstract
\iabstract{
The classic strategy to place a satellite in the geostationary orbit (GEO) relies on chemical propulsion, which involves applying, ideally, only two impulses. This option has proven to be effective and reliable. In recent all-electric satellites the GEO is instead reached with low-thrust transfers. This strategy attains more efficient satellites at the cost of a longer transfer time.
%\st{[Kluever, 2010]}
% * <francesco.topputo@gmail.com> 2016-09-02T13:25:54.596Z:
%
% > %\st{[Kluever, 2010]}
%
% 1) L'abstract e' troppo lungo, provero' ad accorciarlo (non deve essere necessariamente uguale all'extended abstract sottomesso a suo tempo)
%
% ^ <francesco.topputo@gmail.com> 2016-09-02T14:58:35.110Z:
%
% L'ho accorciato. Nota: non c'e' bisogno di usare \textit{GEO} ma semplicemente GEO (non c'e' motivo per scriverlo in corsivo)
%
% ^ <lucaferella23@gmail.com> 2016-09-03T02:50:23.475Z:
%
% Buongiorno. Va bene, modificherò le scritte in corsivo.
%
% ^ <francesco.topputo@gmail.com> 2016-09-04T06:36:29.505Z.
These two options give rise to platforms having divergent features: fully chemical satellites, with short transfer times but large propellant masses, and fully electric satellites, with low propellant mass fractions but long transfer times. This dichotomy forbids widening the trade space in preliminary design of GEO satellites. A way to account for intermediate design solutions consists in allowing the two propulsion systems to coexist on the GEO platform. In principle, a hybrid transfer may lead to a family of design solutions that fill the gap between the two boundary solutions. However, a methodology to preliminary assess this kind of solutions is not established yet, and requires non-trivial procedures. \\
In this paper we elaborate on the concept of hybrid propelled satellites for GEO applications. A preliminary design procedure is derived, which allows evaluating the usefulness of hybrid platforms for given payload. The realistic effect of the solar arrays degradation due to passages through the radiation belts is modeled, and elements of preliminary system design are combined to those of preliminary trajectory optimization. These involve power subsystem sizing, electric and chemical propulsion modeling, and multispiral, long-duration, low-thrust trajectory optimization. The overall benefits of hybrid GEO transfers are evaluated by using economical models as well. The results show that hybrid platforms may represent a viable option to widen the trade space for the next generation of GEO satellites.
}